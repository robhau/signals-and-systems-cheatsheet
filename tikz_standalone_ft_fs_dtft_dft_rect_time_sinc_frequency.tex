\documentclass[paper=a3paper,landscape]{standalone}
\usepackage{trfsigns}
\usepackage{pgfplots}
\usetikzlibrary{calc}
\usepackage{comment}
\usepackage{url}
\usepackage{amssymb}
\usepackage{xfrac}
\usepackage{sig_sys_macros}
\newcommand{\sinc}{\mathrm{sinc}}
\newcommand{\rect}{\mathrm{rect}}
\newcommand{\psinc}{\mathrm{psinc}}
% uncomment for squared paper
%\renewcommand{\paperwidth}{\paperheight}
\begin{document}
	\begin{tikzpicture}[
		declare function = {
			sinc(\t)=sin(deg(\t))/\t*(\t!=0)+1*(\t==0);
			rect(\t)=and(\t>=-1/2,\t<=1/2);
			square(\t)= and(mod(\t,2)>=0,mod(\t,2)<1);
			RECT(\t,\n) = 1*and(\t>=0,\t<\n);
			diric(\w,\n) =( mod(abs(\w),2*pi)==0)*\n +( mod(abs(\w),2*pi)!=0)*sin(deg(\n/2*abs(\w)))/sin(deg(abs(\w)/2));
			diric2(\u,\m,\n)=( mod(abs(\u),\n)==0)*\m +(mod(abs(\u),\n)!=0)*sin(deg(\m/\n*pi*\u))/sin(deg(1/\n*pi*\u));
		},
		]
		%
		% period
		\newcommand{\T}{5}
		% impulse width
		\newcommand{\Th}{2}
		% axis scaling below is hard coded!
		% label font size
		\newcommand{\FONTSIZE}{\scriptsize}
		% time domain color
		\newcommand{\TCOLOR}{C0}
		% frequency domain color
		\newcommand{\FCOLOR}{C1}
		% line width of continuous functions
		\newcommand{\LINEWIDTH}{ultra thick}
		% marker size of discret functions
		\newcommand{\MARKSIZE}{1.6pt}
		\pgfmathsetmacro{\W}{pi}
		\pgfmathsetmacro{\PAPERWIDTHSQUARED}{\paperwidth/32/32*\paperwidth}
		\pgfmathsetmacro{\PAPERHEIGHTSQUARED}{\paperheight/32/32*\paperheight}
		\pgfmathsetmacro{\ANGLE}{atan(\paperheight/\paperwidth)}
		%
		%
		%
		\draw[C3, ultra thick] (0,0) rectangle (\paperwidth,\paperheight);
		%
		%
		%
		% red cross
		%
		\draw[C3,ultra thick] (3/8*\paperwidth,1/2*\paperheight) -- (5/8*\paperwidth,1/2*\paperheight);
		\draw[C3,ultra thick] (1/2*\paperwidth,3/8*\paperheight) -- (1/2*\paperwidth,5/8*\paperheight);
		%
		%
		%
		% continuation of cross lines at the paper border
		%
		\draw[C3,ultra thick] (0,1/2*\paperheight) -- (1/8*\paperwidth,1/2*\paperheight);
		\draw[C3,ultra thick] (7/8*\paperwidth,1/2*\paperheight) -- (\paperwidth,1/2*\paperheight);
		\draw[C3,ultra thick] (1/2*\paperwidth,0) -- (1/2*\paperwidth,1/8*\paperheight);
		\draw[C3,ultra thick] (1/2*\paperwidth,7/8*\paperheight) -- (1/2*\paperwidth,\paperheight);
		%
		%
		%
		% transformation abbreviations in corners
		%
		\node[anchor=north west] at (0,\paperheight) (FT) {\Huge \textbf{FT}};
		\node[anchor=north east] at (\paperwidth,\paperheight) (FS) {\Huge \textbf{FS}};
		\node[anchor=south west] at (0,0) (DTFT) {\Huge \textbf{DTFT}};
		\node[anchor=south east] at (\paperwidth,0) (DFT) {\Huge \textbf{DFT}};
		%
		\node[anchor=north west] at (11/64*\paperwidth,17/20*\paperheight) (whFT) {\FONTSIZE $\omega_h = \frac{2\pi}{T_h}$};
		\node[anchor=north west] at (25/32*\paperwidth,17/20*\paperheight) (woFS) {\FONTSIZE $\omega_0 = \frac{2\pi}{T_0}$};
		\node[anchor=north west] at (25/32*\paperwidth,7/20*\paperheight) (NFFT) {\FONTSIZE $N=10$};
		\node[anchor=north west] at (24.5/32*\paperwidth,6.5/20*\paperheight) (NFFTnote) {\FONTSIZE exact DFT pair};

		\node[anchor=north west] at (11/64*\paperwidth,7/20*\paperheight) (WcDTFT) {\FONTSIZE $\Omega_h = \frac{2\pi}{K_h}$};
		\node[anchor=north west] at (8/64*\paperwidth,6.5/20*\paperheight) (WcDTFTfor) {\FONTSIZE $K_h\in\mathbb{N}$, here for $K_h=5$};


		\node[anchor=north west] at (48.5/128*\paperwidth,2/128*\paperheight) (author) {\tiny Robert Hauser, Frank Schultz, University of Rostock, \url{https://github.com/spatialaudio/signals-and-systems-cheatsheet}, CC BY 4.0, 2024-04-06};
		%
		% FT Plots
		%
		% rect time domain
		\begin{axis}[
			axis lines = middle,
			axis line style = {-latex},
			xlabel = {$t$},
			x label style = {anchor=north},
			ylabel = {$x(t)$},
			y label style={anchor=west},
			width = 18/80*\paperwidth,
			height = 18/80*\paperheight,
			at = {(1/10*\paperwidth,17/20*\paperheight)},
			xtick = {-6,-4,-2,-1,0,1,2,4,6},
			xticklabels = {,,,$\frac{-T_h}{2}$,,$\frac{+T_h}{2}$,,,},
			ytick = {0,0.5,1},
			yticklabels = {},
			ticklabel style = {font=\FONTSIZE},
			grid = major,
			ymin = -0.25,
			ymax = 1.125,
			xmin = -8.5,
			xmax = 8.5,
			anchor = north,
			]
			\addplot[domain=-7.5:7.5,
			samples = 257,
			color=\TCOLOR,
			\LINEWIDTH,]
			{rect(x/\Th)};
			\node[anchor=center] at (axis cs:-\Th/2*1.75,1) {\FONTSIZE $A$};
		\end{axis}
		%
		% rect frequency domain
		%
		\begin{axis}[
			axis lines = middle,
			axis line style = {-latex},
			xlabel = {$\omega$},
			x label style = {anchor=north},
			ylabel = {$X\left(\omega\right)$},
			y label style={anchor=west},
			xtick = {-6*pi,-5*pi,...,6*pi},
			xticklabels = {,$-5\omega_h$,,,,$-\omega_h$,,$+\omega_h$,,,,$5\omega_h$,},
			ytick = {0,1,2},
			yticklabels = {},
			ticklabel style={font=\FONTSIZE},
			grid = major,
			width = 18/80*\paperwidth,
			height = 18/80*\paperheight,
			at = {(3/10*\paperwidth,17/20*\paperheight)},
			ymin = -0.5,
			ymax = 2.25,
			xmin = -17/15*6*pi,
			xmax = 17/15*6*pi,
			anchor = north,
			]
			\addplot[domain=-6*pi:6*pi,
			samples = 129,
			color=\FCOLOR,
			\LINEWIDTH,]
			{\Th*sinc(x*\Th/2)};
			\node[anchor=center] at (axis cs:-2*pi/\Th,2) {\FONTSIZE $A T_h$};
		\end{axis}
		%
		%
		%
		% FS Plots
		%
		% rect time domain
		\begin{axis}[
			axis lines = middle,
			axis line style = {-latex},
			xlabel = {$t$},
			x label style = {anchor=north},
			ylabel = {$x(t)$},
			y label style={anchor=west},
			width = 18/80*\paperwidth,
			height = 18/80*\paperheight,
			at = {(7/10*\paperwidth,17/20*\paperheight)},
			xtick = {-5,-2.5,-2,-1,1,2,2.5,5},
			xticklabels = {$-T_0$,,,$\frac{-T_h}{2}$,$\frac{+T_h}{2}$,,,$+T_0$},
			ytick = {0,0.5,1},
			yticklabels = {},
			ticklabel style={font=\FONTSIZE},
			grid = major,
			ymin = -0.25,
			ymax = 1.125,
			xmin = -8.5,
			xmax = 8.5,
			anchor = north,
			]
			\addplot[domain=-7.5:7.5,
			samples = 257,
			color=\TCOLOR,
			\LINEWIDTH,]
			{rect(x/\Th)+rect((x-\T)/\Th)+rect((x+\T)/\Th)};
			\node[anchor=center] at (axis cs:-2.5,1) {\FONTSIZE $A$};
		\end{axis}
		%
		% rect frequency domain
		%
		\begin{axis}[
			axis lines = middle,
			axis line style = {-latex},
			xlabel = {$\mu$},
			x label style = {anchor=north},
			ylabel = {$\tilde{X}\left[\mu\right]$},
			y label style={anchor=west},
			xtick = {-15,-10,-5,0,5,10,15},
			xticklabels = {,,$-5$,,$+5$,$+10$,},
			ytick = {0,1,2},
			yticklabels = {},
			ticklabel style={font=\FONTSIZE},
			width = 18/80*\paperwidth,
			height = 18/80*\paperheight,
			at = {(9/10*\paperwidth,17/20*\paperheight)},
			grid = major,
			ymin = -0.5,
			ymax = 2.25,
			xmin = -17,,
			xmax = 17,
			anchor = north,
			]
			\addplot[domain=-15:15,
			samples = 31,
			color=\FCOLOR,
			ycomb,
			mark=*,
			mark size = \MARKSIZE,]
			{\Th*sinc(x*pi*\Th/\T)};
			\node[anchor=center] at (axis cs:-5,2) {\FONTSIZE $A T_h$};
			\node[anchor=center] at (axis cs:-7,1) {\FONTSIZE for $\frac{T_h}{T_0}=\frac{2}{5}$};
		\end{axis}
		%
		%
		%
		% DTFT Plots
		%
		% rect time domain
		\begin{axis}[
			axis lines = middle,
			axis line style = {-latex},
			xlabel = {$k$},
			x label style = {anchor=north},
			ylabel = {$x[k]$},
			y label style={anchor=west},
			width = 18/80*\paperwidth,
			height = 18/80*\paperheight,
			at = {(1/10*\paperwidth,7/20*\paperheight)},
			xtick = {-15,-10,-5,-2.5,0,2.5,5,10,15},
			xticklabels = {,,,$\frac{-K_h}{2}$,,$\frac{+K_h}{2}$,,$+10$,},
			ytick = {0,0.5,1},
			yticklabels = {},
			ticklabel style={font=\FONTSIZE},
			grid = major,
			ymin = -0.25,
			ymax = 1.125,
			xmin = -17,
			xmax = 17,
			anchor = north,
			]
			\addplot[domain=-15:15,
			samples = 31,
			color=\TCOLOR,
			ycomb,
			mark=*,
			mark size = \MARKSIZE,]
			{RECT(x+2,5)};
			\node[anchor=center] at (axis cs:-5,1) {\FONTSIZE $A$};
		\end{axis}
		%
		% rect frequency domain
		%
		\begin{axis}[
			axis lines = middle,
			axis line style = {-latex},
			xlabel = {$\Omega$},
			x label style = {anchor=north},
			ylabel = {$X\left(\Omega\right)$},
			y label style={anchor=west},
			width = 18/80*\paperwidth,
			height = 18/80*\paperheight,
			at = {(3/10*\paperwidth,7/20*\paperheight)},
			xtick = {-3*pi,-2*pi,-2*pi/5,0,2*pi/5,2*pi,3*pi},
			xticklabels = {,$-2\pi$,$-\Omega_h$,,$+\Omega_h$,$2\pi$,},
			ytick = {0,2.5,5},
			yticklabels = {},
			ticklabel style={font=\FONTSIZE},
			grid = major,
			ymin = -1.25,
			ymax = 5.625,
			xmin = -17/15*3*pi,
			xmax = 17/15*3*pi,
			anchor = north,
			]
			\addplot[domain=-3*pi:3*pi,
			samples = 129,
			color=\FCOLOR,
			\LINEWIDTH,]
			{diric(x,5)};
			\node[anchor=center] at (axis cs:-pi,5) {\FONTSIZE $A K_h$};
		\end{axis}
		%
		%
		%
		% DFT Plots
		%
		% rect time domain
		\begin{axis}[
			axis lines = middle,
			axis line style = {-latex},
			xlabel = {$k$},
			x label style = {anchor=north},
			ylabel = {$x[k]$},
			y label style={anchor=west},
			width = 18/80*\paperwidth,
			height = 18/80*\paperheight,
			at = {(7/10*\paperwidth,7/20*\paperheight)},
			xtick = {-15,-10,-5,0,5,10,15},
			xticklabels = {,,$-5$,,$+5$,$+10$,},
			ytick = {0,0.5,1},
			yticklabels = {},
			ticklabel style={font=\FONTSIZE},
			grid = major,
			ymin = -0.25,
			ymax = 1.125,
			xmin = -17,
			xmax = 17,
			anchor = north,
			]
			\addplot[domain=-15:15,
			samples = 31,
			color=\TCOLOR,
			ycomb,
			mark=*,
			mark size = \MARKSIZE,]
			{RECT(x+2,5)+RECT(x+2+2*\T,5)+RECT(x+2-2*\T,5)};
			\node[anchor=center] at (axis cs:-5,1) {\FONTSIZE $A$};
		\end{axis}
		%
		% rect frequency domain
		%
		\begin{axis}[
			axis lines = middle,
			axis line style = {-latex},
			xlabel = {$\mu$},
			x label style = {anchor=north},
			ylabel = {$X\left[\mu\right]$},
			y label style={anchor=west},
			width = 18/80*\paperwidth,
			height = 18/80*\paperheight,
			xtick = {-15,-10,-5,0,5,10,15},
			xticklabels = {,,$-5$,,$+5$,$+10$,},
			ytick = {0,2.5,5},
			yticklabels = {},
			ticklabel style={font=\FONTSIZE},
			at = {(9/10*\paperwidth,7/20*\paperheight)},
			grid = major,
			ymin = -1.25,
			ymax = 5.625,
			xmin = -17,
			xmax = +17,
			anchor = north,
			]
			\addplot[domain=-15:15,
			samples = 31,
			color=\FCOLOR,
			ycomb,
			mark=*,
			mark size = \MARKSIZE,
			]
			{diric2(x,5,10)};
			\node[anchor=center] at (axis cs:-5,5) {\FONTSIZE $5 A$};
		\end{axis}
		%
		%
		%
		% relation between FT and FS
		%
		%
		% omega between FT and FS
		%
		\node at (22/40*\paperwidth,6/8*\paperheight) [align=center,color=\FCOLOR] {\Huge $\mathbf{\omega}$};
		%
		% t between FT and FS
		%
		\node at (18/40*\paperwidth,6/8*\paperheight) [align=center,color=\TCOLOR] {\Huge $\mathbf{t}$};
		%
		% open arrow from FT to FS
		%
		% first third
		%
		\draw[C1,thick] (17/40*\paperwidth,25/32*\paperheight) -- (19/40*\paperwidth,25/32*\paperheight);
		%
		% second third (30 ° open)
		%
		% sqrt(3) approx 1.732050808
		%
		\draw[C1,thick] (19/40*\paperwidth,25/32*\paperheight) -- (19/40*\paperwidth+1.732050808/2*2/40*\paperwidth,25/32*\paperheight+1/2*2/40*\paperwidth);
		%
		% last third
		%
		\draw[C1,thick,-latex] (21/40*\paperwidth,25/32*\paperheight) -- (23/40*\paperwidth,25/32*\paperheight) node[above] {Sampling};
		%
		% impulse comb (open arrow)
		%
		\node at (22/40*\paperwidth,27/32*\paperheight) [align=center,color=\FCOLOR] {\Huge $\mathbf{\cdot\Sha}$};
		%
		\node at (18/40*\paperwidth,27/32*\paperheight) [align=center,color=\TCOLOR] {\Huge $\mathbf{\ast\Sha}$};
		%
		% closed arrow from FS to FT
		%
		\draw[C1,thick,latex-] (17/40*\paperwidth,23/32*\paperheight)  node[below] {Interpolation} -- (23/40*\paperwidth,23/32*\paperheight);
		%
		% convolution sinc closed arrow
		%
		\node at (22/40*\paperwidth,21/32*\paperheight) [align=center,color=\FCOLOR] {\huge $\mathbf{\ast\sinc}$};
		%
		% multiplication rect closed arrow
		%
		\node at (18/40*\paperwidth,21/32*\paperheight) [align=center,color=\TCOLOR] {\huge $\mathbf{\cdot\rect}$};
		%
		%
		%
		%
		%
		%
		%
		%
		%
		% relation between DTFT and DFT
		%
		%
		% omega between DTFT and DFT
		%
		\node at (22/40*\paperwidth,2/8*\paperheight) [align=center,color=\FCOLOR] {\Huge $\mathbf{\omega}$};
		%
		% t between DTFT and DFT
		%
		\node at (18/40*\paperwidth,2/8*\paperheight) [align=center,color=\TCOLOR] {\Huge $\mathbf{t}$};
		%
		% open arrow from DTFT to DFT
		%
		% first third
		%
		\draw[C1,thick] (17/40*\paperwidth,9/32*\paperheight) -- (19/40*\paperwidth,9/32*\paperheight);
		%
		% second third (30 ° open)
		%
		% sqrt(3) approx 1.732050808
		%
		\draw[C1,thick] (19/40*\paperwidth,9/32*\paperheight) -- (19/40*\paperwidth+1.732050808/2*2/40*\paperwidth,9/32*\paperheight+1/2*2/40*\paperwidth);
		%
		% last third
		%
		\draw[C1,thick,-latex] (21/40*\paperwidth,9/32*\paperheight) -- (23/40*\paperwidth,9/32*\paperheight) node[above] {Sampling};
		%
		% dirac comb open arrow
		%
		\node at (22/40*\paperwidth,11/32*\paperheight) [align=center,color=\FCOLOR] {\Huge $\mathbf{\cdot\Sha}$};
		%
		\node at (18/40*\paperwidth,11/32*\paperheight) [align=center,color=\TCOLOR] {\Huge $\mathbf{\ast\Sha}$};
		%
		% closed arrow from DFT to DTFT
		%
		\draw[C1,thick,latex-] (17/40*\paperwidth,7/32*\paperheight) node[below]{Interpolation}-- (23/40*\paperwidth,7/32*\paperheight);
		%
		% convolution sinc closed arrow
		%
		\node at (22/40*\paperwidth,5/32*\paperheight) [align=center,color=\FCOLOR] {\huge $\mathbf{\ast\psinc}$};
		%
		% multiplication rect closed arrow
		%
		\node at (18/40*\paperwidth,5/32*\paperheight) [align=center,color=\TCOLOR] {\huge $\mathbf{\cdot\rect}$};
		%
		%
		%
		% relation between FT and DTFT
		%
		%
		% omega between FT and DTFT
		%
		\node at (1/5*\paperwidth,18/40*\paperheight) [align=center,color=\FCOLOR,] {\Huge $\mathbf{\omega}$};
		%
		% t between FT amd FS
		%
		\node at (1/5*\paperwidth,22/40*\paperheight) [align=center,color=\TCOLOR,] {\Huge $\mathbf{t}$};
		%
		% open arrow from FT to DTFT
		%
		% first third
		% 7/32
		\draw[C0,thick] (8/32*\paperwidth,23/40*\paperheight) -- (8/32*\paperwidth,21/40*\paperheight);
		%
		% second third (30 ° open)
		%
		% sqrt(3) approx 1.732050808
		%
		\draw[C0,thick] (8/32*\paperwidth,21/40*\paperheight) -- (8/32*\paperwidth+1/2*2/40*\paperheight,21/40*\paperheight-1.732050808/2*2/40*\paperheight);
		%
		% last third
		%
		\draw[C0,thick,-latex] (8/32*\paperwidth,19/40*\paperheight) -- (8/32*\paperwidth,17/40*\paperheight) node[below]{Sampling};
		%
		% dirac comb open arrow
		%
		\node at (10/32*\paperwidth,18/40*\paperheight) [align=center,color=\FCOLOR,] {\Huge $\mathbf{\ast\Sha}$};
		%
		\node at (10/32*\paperwidth,22/40*\paperheight) [align=center,color=\TCOLOR,] {\Huge $\mathbf{\cdot\Sha}$};
		%
		% closed arrow from DTFT to
		%
		%\draw[black,thick,-latex] (7/32*\paperwidth,17/40*\paperheight) -- (7/32*\paperwidth,23/40*\paperheight);
		%
		% Faltung sinc unter geschlossenem Pfeil
		%
		%\node at (5/32*\paperwidth,18/40*\paperheight) [align=center,color=\FCOLOR,] {\huge $\mathbf{\cdot\rect}$};
		%
		% Multiplikation rect über geschlossenen Pfeil
		%
		%\node at (5/32*\paperwidth,22/40*\paperheight) [align=center,color=\TCOLOR,] {\huge $\mathbf{\ast\sinc}$};
		%
		%
		%
		% relation between FS and DFT
		%
		%
		% omega between FS and DFT
		%
		\node at (4/6*\paperwidth,18/40*\paperheight) [align=center,color=\FCOLOR,] {\Huge $\mathbf{\omega}$};
		%
		% t between FS and DFT
		%
		\node at (4/6*\paperwidth,22/40*\paperheight) [align=center,color=\TCOLOR,] {\Huge $\mathbf{t}$};
		%
		% open arrow from FS to DFT
		%
		% first third
		% 7/32
		\draw[C0,thick] (23/32*\paperwidth,23/40*\paperheight) -- (23/32*\paperwidth,21/40*\paperheight);
		%
		% second third (30 ° open)
		%
		% sqrt(3) approx 1.732050808
		%
		\draw[C0,thick] (23/32*\paperwidth,21/40*\paperheight) -- (23/32*\paperwidth+1/2*2/40*\paperheight,21/40*\paperheight-1.732050808/2*2/40*\paperheight);
		%
		% last third
		%
		\draw[C0,thick,-latex] (23/32*\paperwidth,19/40*\paperheight) -- (23/32*\paperwidth,17/40*\paperheight) node[below]{Sampling};
		%
		% dirac comb open arrow
		%
		\node at (25/32*\paperwidth,18/40*\paperheight) [align=center,color=\FCOLOR,] {\Huge $\mathbf{\ast\Sha}$};
		%
		\node at (25/32*\paperwidth,22/40*\paperheight) [align=center,color=\TCOLOR] {\Huge $\mathbf{\cdot\Sha}$};
		%
		% closed arrow from DFT to FS
		%
		%\draw[black,thick,-latex] (23/32*\paperwidth,17/40*\paperheight) -- (23/32*\paperwidth,23/40*\paperheight);
		%
		% Faltung psinc links von geschlossenem Pfeil
		%
		%\node at (21/32*\paperwidth,18/40*\paperheight) [align=center,color=\FCOLOR,] {\huge $\mathbf{\cdot\rect}$};
		%
		% Multiplikation rect über geschlossenen Pfeil
		%
		%\node at (21/32*\paperwidth,22/40*\paperheight) [align=center,color=\TCOLOR,] {\huge $\mathbf{\ast\psinc}$};
		%
		%
		%
		%
		%
		%
		% arrow from FS to DTFT
		%
		\draw[-latex, ultra thick] (1/2*\paperwidth,1/2*\paperheight) -- node[above,rotate=\ANGLE,color=\TCOLOR,xshift=-1/80*\paperwidth] {\FONTSIZE \rotatebox{+90}{$\curvearrowleft$} Aperiodisation + Sampling}
		node[below,rotate=\ANGLE,color=\FCOLOR,xshift=-1/80*\paperwidth] {\FONTSIZE \rotatebox{+90}{$\curvearrowleft$} Interpolation + Periodisation} (7.75/20*\paperwidth,7.75/20*\paperheight);
		\draw[thick] (1/2*\paperwidth,1/2*\paperheight) -- node[above,rotate=-\ANGLE,color=\TCOLOR,xshift=-1/80*\paperwidth] {\FONTSIZE \rotatebox{+90}{$\curvearrowright$} Periodisation + Sampling}
		node[below,rotate=-\ANGLE,color=\FCOLOR,xshift=-1/80*\paperwidth] {\FONTSIZE  \rotatebox{+90}{$\curvearrowright$} Sampling + Periodisation} (7.75/20*\paperwidth,12.25/20*\paperheight);		%
		%
		% arrow from DTFT to FS
		%
		%\draw[-latex] (1/2*\paperwidth,1/2*\paperheight) -- node[above,rotate=\ANGLE,color=\TCOLOR,xshift=1/80*\paperwidth] {\FONTSIZE Periodic + Interpolation}
		%node[below,rotate=\ANGLE,color=\FCOLOR,xshift=1/80*\paperwidth] {\FONTSIZE Aperiodic + Sampling} (12.25/20*\paperwidth,12.25/20*\paperheight);
		%
		%
		% arrow from FT to DFT
		%
		\draw[-latex, ultra thick] (1/2*\paperwidth,1/2*\paperheight) -- node[above,rotate=-\ANGLE,color=\TCOLOR,xshift=1/80*\paperwidth] {\FONTSIZE Sampling + Periodisation \rotatebox{-90}{$\curvearrowleft$}}
		node[below,rotate=-\ANGLE,color=\FCOLOR,xshift=1/80*\paperwidth] {\FONTSIZE Periodisation + Sampling \rotatebox{-90}{$\curvearrowleft$}} (12.25/20*\paperwidth,7.75/20*\paperheight);
		\draw[thick] (1/2*\paperwidth,1/2*\paperheight) -- node[above,rotate=\ANGLE,color=\TCOLOR,xshift=1/80*\paperwidth] {\FONTSIZE Sampling + Aperiodisation \rotatebox{-90}{$\curvearrowright$}}
		node[below,rotate=\ANGLE,color=\FCOLOR,xshift=1/80*\paperwidth] {\FONTSIZE Periodisation + Interpolation \rotatebox{-90}{$\curvearrowright$}} (12.25/20*\paperwidth,12.25/20*\paperheight);
		%
		%
		% arrow from DFT to FT
		%
		%\draw[-latex] (1/2*\paperwidth,1/2*\paperheight) -- node[above,rotate=-\ANGLE,color=\TCOLOR,xshift=-1/80*\paperwidth] {\FONTSIZE Aperiodic + Interpolation}
		%node[below,rotate=-\ANGLE,color=\FCOLOR,xshift=-1/80*\paperwidth] {\FONTSIZE Aperiodic + Interpolation} (7.75/20*\paperwidth,12.25/20*\paperheight);
		%
		%
		%
		%
		%
		%
		%
		%
		%
		%\draw[black,ultra thick] (0.5/20*\paperwidth,0) -- (0.5/20*\paperwidth,\paperheight);
		%\draw[black,ultra thick] (8/20*\paperwidth,0) -- (8/20*\paperwidth,\paperheight);
		%
		%
		%\draw[black,ultra thick] (19.5/20*\paperwidth,0) -- (19.5/20*\paperwidth,\paperheight);
		%\draw[black,ultra thick] (12/20*\paperwidth,0) -- (12/20*\paperwidth,\paperheight);
		%
		%
		%\draw[black,ultra thick] (0,0.5/20*\paperheight) -- (\paperwidth,0.5/20*\paperheight);
		%\draw[black,ultra thick] (0,8/20*\paperheight) -- (\paperwidth,8/20*\paperheight);
		%
		%
		%\draw[black,ultra thick] (0,12/20*\paperheight) -- (\paperwidth,12/20*\paperheight);
		%\draw[black,ultra thick] (0,19.5/20*\paperheight) -- (\paperwidth,19.5/20*\paperheight);
		%
		%
		\begin{comment}
		\end{comment}
	\end{tikzpicture}
\end{document}
